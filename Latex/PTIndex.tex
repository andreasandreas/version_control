\input{./Parameter/Parameter.tex}
\input{./Parameter/Makros.tex}
\usepackage{gnuplot-lua-tikz}

%-----------------------------------------------opening-----------------------------------------------
\usepackage{enumitem}

%\subject{Notes}
\title{Programming Techniques
\ \\
%\vspace*{2 cm}\includegraphics[scale=0.3]{./Bilder/Titel3.jpg}\vfill
 }
\date{}
\publishers{Andreas Urech\\D-ITET\\ETH
}
%\publishers{Andreas Urech \\
%\ \\
%Copy for Markus Urech }
\begin{document}
\maketitle
\thispagestyle{empty}
\newpage
\thispagestyle{empty}
%\pagenumbering{roman}
\tableofcontents
\newpage
%\pagenumbering{arabic}
%\setcounter{page}{1}
%\setcounter{chapter}{-1}	% Macht erstes Kapitel zu Kapitel 0
%\chapter{Programming Project}
%\begin{center}
%\includegraphics[scale=1]{./Bilder/Tree1.png}
%\captionof{figure}[a]{Project Structure}
%\end{center}
%$\mathbb{\dagger}$  
%Executable main gets saved here. \\
%\ \\
%$\mathbb{\ddag}$
%Source code of the the header files (forward declared classes) and the class implementations get saved here.\\
%$\mathbb{*}$\\
%Executable main and the compiled classes.\\
%$\mathbb{**}$\\
%The header files out of $/$usr$/$include and the libraries out of $/$usr$/$lib get saved here.\\
%\\
%\BF{Library:}\\
%Collection of functions/objects for a specific programm. It helps to build an application.\\
%\ \\
%\BF{API:}\\
%Interface to other programs or even the OS.
%\\
%In embedded systems, when using a simple micro controller there is often times not even an OS. The libraries resp. the API make up 'the OS'.\\
%\ \\
%\BF{Compiler:}\\
%The gnu gcc-compiler compiles c programs and links the c-libraries. The GNU g++ compiler compiles c++ programs and links the c++libraries. However the gcc compiler can be 'morphed' in the c++ compiler. 
%\TLi{ }{./Code/compiler}
%Both lines are equivalent. Considering the second line - The first element is a compiler option, the second two are linker options.\\ 
%\begin{center}
%\includegraphics[scale=0.9]{./Bilder/EclipseAb.png}
%\captionof{figure}[a]{Eclipse}
%\end{center}
%\newpage
%\begin{center}
%\includegraphics[scale=0.44]{./Bilder/EclipseC.png}
%\end{center}
%\TLi{ }{./Code/ios}

\chapter{Make Objects}
The files to be tracked have simply to be mentioned in \TT{add$\_$executable()} CMake takes care of the dependencies. Only the source files \TT{.cpp} have to mentioned, since CMake track the \TT{.hpp} files automatically since they are included in the source-files.
\begin{center}
\tikzset{
	A/.pic={
        code={
\draw[black](0,0)node[rotate=00]{\tiny \texttt{A}};
\draw[black,thick](-0.5,-0.15)rectangle(0.5,0.15);   
							}
	}
}

\tikzset{
	B/.pic={
        code={
\draw[black](0,0)node[rotate=00]{\tiny \texttt{B}};
\draw[black,thick](-0.5,-0.15)rectangle(0.5,0.15);   
							}
	}
}

\tikzset{
	C/.pic={
        code={
\draw[black](0,0)node[rotate=00]{\tiny \texttt{C}};
\draw[black,thick](-0.5,-0.15)rectangle(0.5,0.15);   
							}
	}
}

\tikzset{
	D/.pic={
        code={
\draw[black](0,0)node[rotate=00]{\tiny \texttt{D}};
\draw[black,thick](-0.5,-0.15)rectangle(0.5,0.15);   
							}
	}
}

\tikzset{
	main/.pic={
        code={
\draw[black](0,0)node[rotate=00]{\tiny \texttt{main}};
\draw[black,thick](-0.5,-0.15)rectangle(0.5,0.15);   
							}
	}
}

\begin{tikzpicture}
%Gitter
%\draw[step=0.5cm,very thin,black!20] (-6,-6) grid(6,6);
%\draw(-6,0)--(6,0);
%\draw(0,6)--(0,-6);
\path (0,2.75) pic {main};
\path (-1,2) pic {A};
\path (1,2) pic {B};
\path (-1,1.25) pic {C};
\path (-1,0.5) pic {D};
\draw [black,->] (0,2.6)--(-1,2.15);
\draw [black,->] (0,2.6)--(1,2.15);
\draw [black,->] (-1,1.85)--(-1,1.4);
\draw [black,->] (-1,1.1)--(-1,0.65);
\draw [black,->] (1,1.85)--(-1,0.65);
\end{tikzpicture}

\captionof{figure}[D]{Dependency Structure}
\end{center}
\TwinLs
{\TLi{Excerpt Makefile}{./Code/Makefile4.mk}}
{\TLi{Excerpt CMakeList}{./Code/CMakeLists0.txt}}



\newpage
\section{Makefile}
The make-file must be saved as \TT{Makefile} \footnote{it can be saved with arbitrary in this case it has to be called with:\\ \TT{make -f makefileName}}. It is similar to a bash file. The make-file automatically monitors whether a file got changed and therefore needs new compiling or new linking respectively.
\TLi{General form of a make file}{./Code/MakeFile0}
Only if the files listed in dependencies feature a change the action statements will be executed.\\
Variables can be declared they are accessed with \TT{\$(.)}.\\
The following wild cards are used:\\
$ \begin{array}{ll}
\TT{\$} \ \mathtt{\hat{ }} & \text{all Dependencies} \\
\TT{\$<} & \text{first Dependency} \\ 
\TT{\$@} & \text{name of the target}  
\end{array} 
$\\
The dependency structure expressed by suffixes. All \TT{.s} files depend on the \TT{.c} files.
\TLi{}{./Code/SuffixMake.txt}
\TLi{Example}{./Code/daMake.mk}

By default Makefile targets are file-targets. However sometimes Makefile should run commands, that do not represent physical files in the file-system. These special targets need to be declared, so that make does not look for a file. Targets that are not files are called \TT{PHONY}.
%\TLi{Makefile}{./Code/Makefile3}
\newpage

\section{CMake}\label{sec:CMake}
CMake is a program that creates make files according to the CMakeList.txt specifications.\\
Hence the CMake configurations have to be made in the file \TT{CMakelists.txt}.
\begin{center}
\input{./Bilder/Structure7.tex}
%\captionof{figure}[D]{Typical CMake Project-Structure}
\end{center}
\ \\
The $\dagger$ directories are created by the makefile as configured in listing (\ref{Main CMakeList.txt}). CMakeList-files can be sprinkled everywhere desired, we preferably going downwards

\TLi{Main CMakeList.txt}{./Code/CMakeListsMain.txt}
Some Build-types:\\
$\begin{array}{lll}
\TT{Debug} &\TT{=} & \TT{g++ -g}\\ 
\TT{Release} &\TT{=} & \TT{g++ -O3 -DNDEBUG}\\
\TT{None} &\TT{=} & \TT{g++}\\
\end{array}$



\newpage
\TLi{Src  CMakeList.txt}{./Code/CMakeListsSrc.txt}
\TLi{Test CMakeList.txt}{./Code/CMakeListsTest.txt}

%\TLi{Cross Platform Generic CMakeLists.txt}{./Code/CMakeLists11.txt}
%\TLi{Additional CMakeLists.txt in Lib-dir }{./Code/CMakeLists9.txt}

The files to be tracked simply have to be mentioned in \TT{add$\_$executable()} CMake takes care of the dependencies. Only the source files \TT{.cpp} have to mentioned, since CMake track the \TT{.hpp} files automatically since they are included in the source-files.
%\BF{CMakeLists.txt}
%\TLi{Generic CMakeLists.txt}{./Code/CMakeLists2.txt}
\TT{include$\_$directories} adds the directory to the list of directories the pre-compiler searches for files.


The CMakeVariable \TT{CMAKE$\_$SOURCE$\_$DIR} contains the address of the directory where the file \TT{CMakeLists.txt} is found. % (\TT{\$...} $/$\TT{Src}).
%\newpage
\TLi{Create a Make File with Cmake}{./Code/CMakeListsExample1.txt}
The parameter given to \TT{cmake} refers to the folder containing \TT{CMakeLists.txt}. Since it is located in one higher hierarchy it holds ..$/$\TT{Src}.
The makefile and some cmake files get created in the Built-directory.
The generator\footnote{Generator meaning make and g++} creating the acutal objects and executable can be chosen explictially on calling cmake. 
\TLi{Create a Make File with Cmake}{./Code/CMakeListsExample2.txt}

\newpage
\subsection{Testing with CMake}
CMake allows to create a test-executables. A passed test is assumed to return \TT{0}. Each test is an entire executable, meaning each test is created in a separate \TT{main}-function. Tester-Examples:\\
\TwinLs
{\CppLi{GenomeTest1.cpp Example}{./Code/GenomeTest1.cpp}}
{\CppLi{GenomeTest2.cpp Example}{./Code/GenomeTest2.cpp}}
The test are started by \TT{make all test}.




\subsection{Plattform specific Commands}
The \TT{CMakeLists.txt} can contain platform specific commands. 
%\TLi{UNIX}{./Code/CMakeLists3.txt}
In order to keep the \TT{CMakeLists.txt} plattform independent the platform can be discoverd, and the respecitve commands can be issued correctly.
\TLi{Detect the OS Compiler}{./Code/CMakeLists5.txt}
\TLi{Plattform specific}{./Code/CMakeLists6.txt}
\newpage







%#http://cpprocks.com/using-cmake-to-build-a-cross-platform-project-with-a-boost-dependency/
%\subsection{CMakeLists.txt}
%\TLi{Generic CMakeLists.txt}{./Code/CMakeListsGeneric.txt}
%\newpage
%\subsubsection{Example:}
%\TLi{CMakeLists.txt}{./Code/CMakeLists.txt}
%\TLi{Create a Make File with Cmake}{./Code/CMakeListsExample.txt}
%$CDT4$ supports CDT 4.0 or higher.
%The last parameter refers to the source files, since they are in the $src$ folder located in one higher hierarchy it holds ..$/$src. \\
%For the Makefile that was generated see (\ref{sec:Makefile}).


\chapter{IDE}
Integrated Development Environment, e.g. Eclipse, provide additional tools, notably debugger, automatic makefile generation etc.
\section{Eclipse}
\BF{Workspace:}\\
Where the platform an all the installed plug-ins, config-files, tmp-files etc are stored. The Workspace can contain multiple projects.\\
The projects (meaning the \TT{C} code) are not necessarily physically located in the workspace path. In the workspace they can be simple pointers to other locations on the disk.
\ \\
\BF{Project:}\\
Section


\section{Creating a Project from bare source files}
%\chapter{GIT}
\TT{GIT} is a Version Control Software. GitHub, GitLab, GitBucket etc. are hosting repository services.
\TLi{Setting up GitLab}{./Code/GITLABsetup.txt}
\section{Setting up GIT}
\TLi{Setting up Git:}{./Code/GITsetup.txt}
%\BF{clone:}\\
\BF{Cloning:}\\
Clone just means to get a copy of the repository on the local machine.
\TLi{clone}{./Code/GITclone.txt}
\BF{Making a repo of an existing local folder:}\\
For existing local folders the procedure to get a remote repository is as follows. First the empty remote repository (project) \TT{origin/master} has to be created on the git-web-interface. Then the existing local folder is included via shell.
\TLi{remote add}{./Code/GITremoteadd.txt} 
In the file \TT{.gitignore} it can be specified which filse are to be ignored by git.
\TLi{}{./Code/GitIgnore.txt}
\section{GIT-Handling}
\TLi{GIT Manual}{./Code/GIThelp.txt}
%\BF{add-commit-push:}\\
Before a file or a folder can be committed it has to be moved to the staging area this is done with \TT{add}. The \TT{commit}-command saves the changes\footnote{\TT{commit} is like a save button, it saves a state to which one could fall back} together with a commit-hash. The \TT{push}-command pushes the files/folders in the staging area to \TT{origin} on \TT{master}, where\TT{origin} is the remote repository and \TT{master} is the current branch.  
\TLi{add commit push}{./Code/GITacp.txt}
%\BF{rm}
\TLi{rm}{./Code/GITrm.txt}
%\BF{log}
\TT{log} lists all the commits, with the commit message and the commit-hash.
\TLi{log}{./Code/GITlog.txt}

\TLi{diff}{./Code/GITdiff0.txt}
\TLi{diff example}{./Code/GITdiff1.txt}
\TLi{blame}{./Code/GITblame.txt}

\BF{pull:}\\
\TT{pull} up-dates the one's local repository.
\TLi{pull}{./Code/GITpull.txt}

\BF{Reset to a certain commit:}\\
\TT{--hard h1} sets the current branch back and rewrites the history of the branch, commits done after \TT{h1} will no longer be in the branch-history.\\
HEAD points to the current commit, so with \TT{git reset --soft HEAD@\{1\}} HEAD is set back, and the history is not lost.
\TLi{Reset}{./Code/GITreset0.txt}
Alternatively it can be branched of a certain commit see listing \ref{Branch of commit}.

\subsection{Conflicts}
User A pulls \TT{file.txt} from the repository. Then \TT{file.txt} gets changed in the repo by user B. On pushing \TT{file.txt} to the repo by user A a conflict can occur. \\
In order to resolve the conflict, user A has to pull the modified repo $\rightarrow$ the changes of A remain but the commits of B are included too. Conflicts are marked with \TT{<<<<<<<...>>>>>>>} and the respective commit-hash.
\TLi{file.txt with a conflict (after pull by A)}{./Code/file.txt}
Alternatively \TT{git -log} and \TT{diff} can be used in order to find the differences to the repo resp. the conflicts by hand.
\TLi{Differences to the repo}{./Code/GITconflictbyhand.txt}
\newpage
\subsection{Branches and Tags}
\TT{origin} is the remote repository and \TT{master} is the standard name for the main branch.
\begin{center}

\tikzset{
	Commit1/.pic={
        code={
\draw[black,fill=gray, opacity=0.7] (0,0) circle (0.08cm);
\draw[black,right](-0.02,0)node[rotate=90]{\tiny \texttt{commit}};
							}
	}
}

\tikzset{
	Merge/.pic={
        code={
\draw[black,fill=gray, opacity=0.9] (0,0) circle (0.08cm);
\draw[black,left](0,0)node[rotate=90]{\tiny \texttt{merge}};
							}
	}
}

\tikzset{
	Commit2/.pic={
        code={
\draw[black,fill=gray, opacity=0.7] (0,0) circle (0.08cm);
\draw[black,left](-0.02,0)node[rotate=90]{\tiny \texttt{commit}};
							}
	}
}


\tikzset{
	Tag1/.pic={
        code={
\draw[black](-0.15,0.2)rectangle(0.15,1.2);
\draw[black](0,0.7)node[rotate=90]{\tiny \texttt{Tag V1}};
\draw[black,fill=gray, opacity=0.7] (0,0) circle (0.08cm); 
							}
	}
}

\tikzset{
	Tag2/.pic={
        code={
\draw[black](-0.15,0.2)rectangle(0.15,1.2);
\draw[black](0,0.7)node[rotate=90]{\tiny \texttt{Tag V2}};
\draw[black,fill=gray, opacity=0.7] (0,0) circle (0.08cm); 
							}
	}
}


\tikzset{
	Tag3/.pic={
        code={
\draw[black](-0.15,-1.2)rectangle(0.15,-0.2);
\draw[black](0,-0.7)node[rotate=90]{\tiny \texttt{Tag V3}};
\draw[black,fill=gray, opacity=0.7] (0,0) circle (0.08cm); 
							}
	}
}


\begin{tikzpicture}
%Gitter
%\draw[step=0.5cm,very thin,black!20] (-6,-6) grid(6,6);
%\draw(-6,0)--(6,0);
%\draw(0,6)--(0,-6);

\path (-1.5,0.5) pic {Commit1};
\path (-0.5,0.5) pic {Commit2};
\path (1,1) pic {Commit1};



\path (3,0.5) pic {Merge};
\path (-2,0.5) pic {Tag1};
\path (2,0.5) pic {Commit1};
\path (3.5,1) pic {Commit1};
\path (3.5,0.5) pic {Commit2};

\path (-1,0.8) pic {Commit1};
\path (0,1) pic {Tag2};

\path (0,0.2) pic {Commit2};
\path (1,0) pic {Commit2};

\path (1.5,-0.3) pic {Commit2};
\path (2.5,-0.5) pic {Tag3};
\draw [black,ultra thick,->] plot [smooth, tension=0.8] coordinates {(-4,0.5) (5,0.5)};
\draw [black,->] plot [smooth, tension=0.8] coordinates {(1,0) (1.8,-0.4)(3,-0.5)};

\draw [black,thick] plot [smooth, tension=0.8] coordinates {(-1.5,0.5) (-0.7,0.9)(0.5,1)};

%\draw [black,thick,->] plot [smooth, tension=0.8] coordinates {(3,0.5) (3.8,0.9)(5,1)};
\draw [black,thick,->] plot [smooth, tension=0.8] coordinates {(0.5,1) (4,1)};


\draw [black,thick,<-] plot [smooth, tension=0.8] coordinates {(3,0.5) (2,0.9)(0.5,1)};


\draw [black,thick,->] plot [smooth, tension=0.8] coordinates {(-0.5,0.5) (0.3,0.1)(1.5,0)};

%\draw[black,right](0.5,1.15)node[rotate=00]{\tiny \texttt{Branch 1}};
\draw[black,right](4,1)node[rotate=00]{\tiny \texttt{Branch 1}};
\draw[black,right](0.5,0.65)node[rotate=00]{\tiny \texttt{Master}};
\draw[black,right](1.5,0)node[rotate=00]{\tiny \texttt{Branch 2}};
\draw[black,right](3,-0.5)node[rotate=00]{\tiny \texttt{Branch 2.1}};
\draw[black,right](5,0.5)node[rotate=00]{\tiny \texttt{Master}};
%\draw[black,right](4,0.5)node[rotate=00]{\small \texttt{Master}};
\draw[black,right](-4.1,2.2)node[rotate=00]{\small \texttt{Origin:}};
\end{tikzpicture}

\captionof{figure}[b]{Branches and Tags}
\end{center}

\subsubsection{Branching}
\TLi{Create Branch}{./Code/GITbranch.txt}
Branching of a certain commit.
\TLi{Create Branch of commit}{./Code/GITBranchofcommit.txt} 

\TT{branch -d} deletes a branch. After a branch is deleted its commits are still available.
\TLi{Delete Branch}{./Code/GITdelete.txt}

Depending on branch chosen by \TT{checkout} the local repo changes.
\TLi{Checkout a Branch}{./Code/GITcheckout.txt} 


\TT{Merge} takes all the commits (resp. all the changes) of the branch and puts them to current branch. After a merge the branch keeps in existence as an independent branch.


\TLi{Merge}{./Code/GITmerge.txt}

\newpage
\subsubsection{Tags}
Tags are similar to branches, they mark special points on the 'version-tree' releases, important backups etc.\footnote{Tags are displayed under \TT{Tags} (GitLab) or \TT{releases} (GitHub) on the remote repository}Lightweight tags are just special commits, annotated tags have a corresponding message and details about the tagger are saved.
\TLi{Create Tag}{./Code/GITtag0.txt}
\TLi{Create Tag from previous commit}{./Code/GITtag2.txt}
\TLi{Remove Tags}{./Code/GITtag1.txt}

Tags cannot be checked out. However a branch can be created from the tag, an this tag can be checked out.
\TLi{checkout from Tag}{./Code/GITtag3.txt}

%\TLi{Remove Tags}{./Code/GITtag2.txt}
\subsection{Forks}
Collaborators to a repository can make changes without issuing pull requests, others can contribute to the repo via pull-requests.\\
\ \\
A fork creates a new instance of a repository at the point where it currently is. The owner of the forked version can do with it whatever  he wants to. The forked version is not linked to the original version, only in the sense that the fork owner can issue pull-request. 
\TLi{fork}{./Code/GITfork.txt}




A pull-request is a request from fork-owner N with no rights, the original-owner R. N to R: \textit{I have a suggestion, would you like to pull from my repository?} R can grant or deny this suggestion. The pull-request is best done on the web-interface via the pull-request button.
\chapter{GNU-Plots}
Set the plot-script accordingly to generate plot files (png, tikz-files). Call the gnu plot-script with \TT{gnuplot plotScript.plt}.
\TLi{create png}{./Code/plotScript1.plt}
\TLi{create tikz-file}{./Code/plotScript2.plt}
\begin{center}
\input{./Code/cacheEffects.txt}
\end{center}
\TT{using 1:2} descripes the dataset meaning 1 dataset where each row has 2 columns.





\appendix
\chapter{Appendix}


%\chapter{Miscellaneous}
%\section{Namespace}
%Name spaces allow to divide to code into different scopes. Name spaces can be opened at arbitrary places.

%\TwinLs
%{\CppLi{Definition of a name space}{./Code/NameSpace1.cpp}}
%{\CppLi{Access to name space}{./Code/NameSpace2.cpp}}
%An alternative to the scope operator would be the keyword \textit{using} (but it is generally seen as a bad habit).  
%\ \\
%\ \\
%Example:
%\CLi{istream}{./Code/istream}


%\chapter{Open MP Library}
%The open-MP library is a standard library that is included in the basic version of the gcc - compiler. It supports parallel computing in C and Cpp.
%\CLi{OpenMP Example}{./Code/openMP1.c}
%\TLi{Output openMP}{./Code/outputOpenMP1.txt}
%\BF{Header file $omp.h$}
%\TLi{open-MP Header}{./Code/ompPath.txt}
%\TLi{Header File omp.h}{./Code/omp.h}
\section{Cheat-sheet Open MP}

\includegraphics[scale=0.78]{./Bilder/OpenMP1.pdf}
%\captionof{figure}[open MP cheat]{open MP cheat}

\includegraphics[scale=0.78]{./Bilder/OpenMP2.pdf}
%\captionof{figure}[open MP cheat]{open MP cheat}

\includegraphics[scale=0.78]{./Bilder/OpenMP3.pdf}
%\captionof{figure}[open MP cheat]{open MP cheat}

\includegraphics[scale=0.78]{./Bilder/OpenMP4.pdf}




\section{Makefile}\label{sec:Makefile}
This is the makefile that was created by \textit{cmake} according to \textit{CMakeLists.txt}. (see \ref{sec:CMake})
\TLi{Makefile created by cmake}{./Code/Makefile}

\end{document}
